%%=============================================================================
%% Voorwoord
%%=============================================================================

\chapter*{Voorwoord}
\label{ch:voorwoord}

%% TODO:
%% Het voorwoord is het enige deel van de bachelorproef waar je vanuit je
%% eigen standpunt (``ik-vorm'') mag schrijven. Je kan hier bv. motiveren
%% waarom jij het onderwerp wil bespreken.
%% Vergeet ook niet te bedanken wie je geholpen/gesteund/... heeft

Mijn interesse in mobiele- en web applicaties was voor mij de reden om de opleiding Toegepaste Informatica aan de Hogeschool Gent te starten. Deze bachelorproef vormt het sluitstuk in mijn opleiding en de keuze van het onderwerp is tot stand gekomen door de samenwerking met mijn stageplaats Pridiktiv.care - Into.care. Er was nood aan een onderzoek naar offline data opslag en synchronisatie bij hun mobiele applicatie. Het onderwerp ligt in lijn met mijn persoonlijke- en professionele interesses dus was het vanzelfsprekend om met het probleem aan de slag te gaan.

Dankzij de begeleiding van bepaalde personen ben ik er in geslaagd om een interessante en relevante bachelorproef te schrijven. Eerst en vooral wil ik mijn co-promotor en stage-mentor Sam Verschueren bedanken voor zijn inzet en geduld bij de talloze vragen die ik het gesteld in verband met web applicaties. Daarnaast ben ik ook zeer dankbaar voor de intense begeleiding en feedback die ik heb ontvangen van mijn promotor Stefaan De Cock. Tenslotte wil ook mijn partner en vrienden bedanken voor de hulp die ze hebben aangeboden bij het lezen van mijn bachelorproef en de morele ondersteuning.