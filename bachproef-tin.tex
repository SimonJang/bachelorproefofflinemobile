%%=============================================================================
%% LaTeX sjabloon voor bachelorproef, HoGent Bedrijf en Organisatie
%% Opleiding Toegepaste Informatica
%%=============================================================================

\documentclass[fleqn,a4paper,12pt]{book}
\usepackage{graphicx}
\usepackage{listings}
\usepackage{color}
\definecolor{lightgray}{rgb}{.9,.9,.9}
\definecolor{darkgray}{rgb}{.4,.4,.4}
\definecolor{purple}{rgb}{0.65, 0.12, 0.82}

\lstdefinelanguage{JavaScript}{
  keywords={typeof, new, true, false, catch, function, return, null, catch, switch, var, if, in, while, do, else, case, break},
  keywordstyle=\color{blue}\bfseries,
  ndkeywords={class, export, boolean, throw, implements, import, this},
  ndkeywordstyle=\color{darkgray}\bfseries,
  identifierstyle=\color{black},
  sensitive=false,
  comment=[l]{//},
  morecomment=[s]{/*}{*/},
  commentstyle=\color{purple}\ttfamily,
  stringstyle=\color{red}\ttfamily,
  morestring=[b]',
  morestring=[b]"
}
\lstset{
   language=JavaScript,
   backgroundcolor=\color{lightgray},
   extendedchars=true,
   basicstyle=\footnotesize\ttfamily,
   showstringspaces=false,
   showspaces=false,
   numbers=left,
   numberstyle=\footnotesize,
   numbersep=9pt,
   tabsize=2,
   breaklines=true,
   showtabs=false,
   captionpos=b
}
\input{structure}

%%---------- Documenteigenschappen --------------------------------------------
%% TODO: Vul dit aan met je eigen info:

% Je eigen naam
\newcommand{\student}{Simon Jang}

% De naam van je promotor (lector van de opleiding)
\newcommand{\promotor}{Stefaan De Cock}

% De naam van je co-promotor. Als je promotor ook je opdrachtgever is en je
% dus ook inhoudelijk begeleidt (en enkel dan!), mag je dit leeg laten.
\newcommand{\copromotor}{Sam Verschueren}

% Indien je bachelorproef in opdracht van/in samenwerking met een bedrijf of
% externe organisatie geschreven is, geef je hier de naam. Zoniet laat je dit
% zoals het is.
\newcommand{\instelling}{Pridiktiv.care - Into.care}

% De titel van het rapport/bachelorproef
\newcommand{\titel}{Opslag en synchronisatie van offline data bij mobiele applicaties}

% Datum van indienen (gebruik telkens de deadline, ook al geef je eerder af)
\newcommand{\datum}{27 mei 2017}

% Academiejaar
\newcommand{\academiejaar}{2016-2017}

% Examenperiode
%  - 1e semester = 1e examenperiode => 1
%  - 2e semester = 2e examenperiode => 2
%  - tweede zit  = 3e examenperiode => 3
\newcommand{\examenperiode}{2}

%%=============================================================================
%% Inhoud document
%%=============================================================================

\begin{document}

%---------- Taalselectie ------------------------------------------------------
%% Als je je bachelorproef in het Engels schrijft, haal dan onderstaande regel
%% uit commentaar. Let op: de tekst op de voorkaft blijft in het Nederlands, en
%% dat is ook de bedoeling!
%\selectlanguage{english}

%---------- Titelblad ---------------------------------------------------------
\inserttitlepage

%---------- Samenvatting, voorwoord -------------------------------------------
\usechapterimagefalse
%%=============================================================================
%% Samenvatting
%%=============================================================================

%% TODO: De "abstract" of samenvatting is een kernachtige (~ 1 blz. voor een
%% thesis) synthese van het document.
%%
%% Deze aspecten moeten zeker aan bod komen:
%% - Context: waarom is dit werk belangrijk?
%% - Nood: waarom moest dit onderzocht worden?
%% - Taak: wat heb je precies gedaan?
%% - Object: wat staat in dit document geschreven?
%% - Resultaat: wat was het resultaat?
%% - Conclusie: wat is/zijn de belangrijkste conclusie(s)?
%% - Perspectief: blijven er nog vragen open die in de toekomst nog kunnen
%%    onderzocht worden? Wat is een mogelijk vervolg voor jouw onderzoek?
%%
%% LET OP! Een samenvatting is GEEN voorwoord!

\IfLanguageName{english}{%
\selectlanguage{dutch}
\chapter*{Samenvatting}
\lipsum[1-4]
\selectlanguage{english}
}{}

%%---------- Samenvatting -----------------------------------------------------
%%
%% De samenvatting in de hoofdtaal van het document

\chapter*{\IfLanguageName{dutch}{Samenvatting}{Abstract}}

Mobiele applicatieontwikkeling beschouwt toegang tot internet vaak als vanzelfsprekend wanneer een mobiele applicatie wordt gebruikt. Dit is echter niet het geval in elke (werk)omgeving. Daarom moet de ontwikkelaar ervoor zorgen dat de applicatie de data ook lokaal kan opslaan indien de wijzigingen niet meteen kunnen worden doorgevoerd naar de achterliggende infrastructuur. Dit probleem vormt voor ontwikkelaars een uitdaging omdat de data integriteit moet worden gewaarborgd wanneer het toestel terug verbonden is met internet. Het gebruik van een performante en betrouwbare methode voor de data die offline wordt ingegeven te synchroniseren met de online cloud-based databank is dus essentieel. Hierbij is het belangrijk om een onderscheid te maken tussen use cases waarbij de developer gebruik maakt van fully-managed databases zoals onder andere DynamoDB van Amazon Web Services en use cases waarbij men zelf de database beheert op verschillende virtual machines. In het onderzoek is synchronisatie specifiek onderzocht voor de fully-managed database DynamoDB die gebruikt van een serverless microservices architectuur en een Angular applicatie als client applicatie. 

Na onderzoek is gebleken dat er verschillende manieren van synchronisatie zijn. Bij de synchronisatie van de offline data is het belangrijk om de verschillende use cases onder te verdelen volgens synchronisatie methode en zo verder te werken. Deze methodologie werd ook gehanteerd bij de opbouw van van het prototype. Er zijn scenario's waar er geen conflicten zijn maar performantie de belangrijkste factor is bv. Read-Only Optimised. Wanneer er conflicten optreden kan men dan kiezen voor Last Write Wins waarbij er dan onvermijdelijk data verloren gaat of andere strategie\"en van conflict resolution. Terwijl het onderzoek zich voornamelijk focust op data synchronisatietechnieken, kan het interessant zijn om bijvoorbeeld de kost in kaart te brengen om over stappen van een fully-managed database naar een andere provider die synchronisatie toelaat zonder dat men zelf veel rekening moet houden met synchronisatie.
%%=============================================================================
%% Voorwoord
%%=============================================================================

\chapter*{Voorwoord}
\label{ch:voorwoord}

%% TODO:
%% Het voorwoord is het enige deel van de bachelorproef waar je vanuit je
%% eigen standpunt (``ik-vorm'') mag schrijven. Je kan hier bv. motiveren
%% waarom jij het onderwerp wil bespreken.
%% Vergeet ook niet te bedanken wie je geholpen/gesteund/... heeft

Mijn interesse in mobiele- en web applicaties was voor mij de reden om de opleiding Toegepaste Informatica aan de Hogeschool Gent te starten. Deze bachelorproef vormt het sluitstuk in mijn opleiding en de keuze van het onderwerp is tot stand gekomen door de samenwerking met mijn stageplaats Pridiktiv.care - into.care. Er was nood aan een onderzoek naar offline data opslag en synchronisatie bij hun mobiele applicatie. Met de groei van IoT en de digitalisering binnen verschillende sectoren, lijkt offline en online synchronisatie relevanter dan ooit.

Het schrijven van een bachelorpaper is geen eenvoudige opdracht en ik zou daarom enkele personen willen bedanken voor hun ondersteuning en expertise. Eerst en vooral wil ik mijn co-promotor en stagementor Sam Verschueren bedanken voor zijn inzet en geduld bij de talloze vragen die ik het gesteld in verband met webontwikkeling. Daarnaast ben ik ook zeer dankbaar voor de intense begeleiding en feedback die ik heb ontvangen van mijn promotor Stefaan De Cock. Tenslotte wil ook mijn partner en vrienden bedanken voor de hulp die ze hebben aangeboden bij het lezen van mijn bachelorproef en de morele ondersteuning.

%---------- Inhoudstafel ------------------------------------------------------
\pagestyle{empty} % No headers
\tableofcontents % Print the table of contents itself
\cleardoublepage % Forces the first chapter to start on an odd page so it's on the right
\pagestyle{fancy} % Print headers again

%---------- Lijst afkortingen, termen -----------------------------------------
%% Als je een lijst van afkortingen of termen wil toevoegen, dan hoort die
%% hier thuis. Gebruik bijvoorbeeld de ``glossaries'' package.
%%=============================================================================
%% Glossarium
%%=============================================================================

\chapter{Glossarium}
\label{ch:glossarium}

%% TODO: Woordenlijst met alle begrippen

\begin{description}
\item[Single-Page Application, SPA]: Een single-page application (SPA) is een web applicatie of website waarbij noodzakelijke HTML, CSS en JavaScript dynamisch wordt ingeladen of bij de eerste page load. De volledige pagina wordt bij een SPA nooit volledig herladen. Het is wel mogelijk dat onderdelen van de pagina dynamisch wordt gewijzigd. De data van de SPA wordt dynamisch opgevraagd van de web server. Met SPA is mogelijk een om effici\"ent om te springen met de beschikbare bandbreedte.
\item[Angular CLI]: Angular CLI is een command line interface tool waarbij een volledige Angular project wordt gebouwd met minimale configuratie. Er wordt een basis structuur, tests, root module en root component aangemaakt. Met behulp van Webpack worden alle files dan gebundeld in enkele static files. Met Angular CLI wordt er heel wat tijd uitgespaard omdat een groot deel van de configuratie wegvalt. Angular CLI wordt gebruikt de business case van Pridiktiv.
\item[Progressive enhancement]: Progressive enhancement is een ontwikkelstrategie bij web development waarbij de focus wordt gelegd op de belangrijkste business requirements die de web applicatie of website moet invullen. Afhankelijk van de browser en de connectie van de eindgebruiker, kunnen er 'lagen' van functionaliteit (features) worden toegevoegd aan de applicatie of website. Met deze strategie kan een website of web applicatie zich aanpassen aan de eindgebruiker en altijd een basisfunctionaliteit garanderen.
\item[Single Source Of Truth, SSOT]: In de context van het ontwerp van informatica systemen is de single source of truth een techniek om data op een bepaalde manier te structureren zodat die niet gedupliceerd wordt binnen de applicatie en alle referenties verwijzen naar dezelfde 'bron'. Alle informatie wordt opgehaald in een centraal punt en dat is voor de applicatie en haar componenten de enige plaats waar die informatie kan worden opgehaald. De ngrx/redux store vervult die rol in een Angular applicatie.
\item[Conflict resolution]: Een applicatie kan data van een remote databank lokaal opslaan voor offline functionaliteit aan te bieden aan de gebruiker. Wanneer de applicatie terug online gaat en de applicatie of remote databank een verschil opmerkt, dan is er sprake van een conflict. Conflict resolution duidt op de methode(s) die worden gebruikt voor het synchroniseren van de data tussen de verschillende databanken.
\item[Serverless computing]: Een andere naam voor Function as a Service (FaaS) is een cloud-computing executie model waarbij de cloud provider het opstarten en stoppen van een functie volledig zelf beheert. Het opstarten van een functie is op basis van een event. een mogelijk event is bijvoorbeeld een API call naar een HTTP endpoint van een functie. Met serverless computing hoeft een developer zich niet bezig te houden met het configureren en beheren van verschillende servers of virtual machines. Daarnaast spelen schaalbaarheid en kosteneffici\"entie ook een belangrijke rol bij serverless computing. Als FaaS-gebruiker betaal je enkel voor de executietijd en wanneer er meer rekencapaciteit nodig is, gebeurt de schaling volledig automatisch.
\item[ReactiveX]: Reactive Extensions is een verzameling van operatoren die imperatieve\footnote{Imperatieve programmeertalen kunnen de state van een applicatie veranderen. JavaScript is een voorbeeld van een imperatieve programmeertaal} programmeertalen toelaten om een datasequentie te verwerken zonder rekening te houden met het (a)synchrone karakter van de data.
\item[Event-driven Architectuur]: Een software architectuur waarbij de verschillende componenten events genereren, detecteren en op een gepaste manier reageren. De event-driven architectuur vormt de basis van ReactiveX en serverless computing.
\item[Asynchroon]: Asynchroon of 'Asynchronous' in de context van web development wil zeggen dat een bepaalde handeling niet real-time maar periodiek van aard is. Dit is belangrijk wanneer gebruikers geen stabiele verbinding hebben tot een netwerk, wanneer men optimaal gebruik wenst te maken van de beschikbare bandbreedte of eenvoudigweg bij het uitvoeren van een HTTP request. Een voorbeeld is RxJS dat gebaseerd is op de ReactiveX library en volledig gebouwd rond asynchroon programmeren.
\item[Polling]: Bij polling (of pulling) controleert een applicatie met een vast interval of er op een invoer -of uitvoerapparaat nieuwe data beschikbaar is. Het interval of frequentie wanneer de applicatie polling uitvoert, noemt de pollfrequentie. Polling wordt steeds ge\"initialiseerd door de ontvanger van de informatie. Te frequente polling kan een negatieve impact hebben op de performantie van de databron.
\item[Pushing]: Bij push technologie wordt de request voor een datatransactie ge\"initialiseerd door de server of 'publisher' van de informatie via internet.
\item[Microservices Architecture]: Microservices is een architectuur voor het opstellen van server-side enterprise applicaties. Daarbij worden alle onderdelen opgebouwd als een set van losgekoppelde en samenwerkende services. Elke service heeft een beperkte functionaliteit en een beperkte verantwoordelijkheid die kan worden aangeroepen door andere microservices. Dankzij microservices is het mogelijk om om de functionaliteit van een monolitische backend op te delen in verschillende kleine services.
\end{description}

%%---------- Kern -------------------------------------------------------------

%%=============================================================================
%% Inleiding
%%=============================================================================
\chapter{Inleiding}
\label{ch:inleiding}
Het onderzoek zal verschillende methodes voor offline opslag en synchronisatie analyseren en onderzoeken. Het onderzoek zal van elke methode de voor- en nadelen overlopen en de toepassing tonen met behulp van een prototype. In het eerste deel worden de verschillende relevante termen overlopen. In 'Stand van zaken' komt de context en noodzaak van het onderzoek aan bod. In de sectie 'Business Case' wordt de business case van Pridiktiv.case - Into.case toegelicht. Tenslotte volgt de probleemstelling, de onderzoeksvraag en de opzet van de bachelorproef.
%% literatuurstudie. Vergeet niet telkens je bronnen te vermelden!
\section{Terminologie}
\label{sec:terminologie}
In deze sectie komen de verschillende componenten van het onderzoek, het prototype en tools aan bod. Een volledige lijst met begrippen en termen kan u terugvinden onder het hoofdstuk 'Glossarium'. Er is een minimum kennis in verband met software ontwikkeling vereist van de lezer om deze sectie volledig te begrijpen.
\subsection{test}
\section{Business Case}
\label{sec:business-case}
De business case van Pridiktiv vormt het vertrekpunt van deze studie en wordt nu nog wat verder toegelicht. De applicatie draait op een smartphone in een woonzorgcentrum. Het woonzorgcentrum beschikt over draadloos internet maar de dekking is niet het volledige gebouw. Hierdoor moet de applicatie ook offline werken. Wanneer verschillende waarden worden geregistreerd met de applicatie (zoals bloeddruk, gewicht, inname medicatie) worden die offline opgeslagen in een Indexed DB. Wanneer het toestel terug online komt, moeten deze waarden worden doorgestuurd naar de backend. Het is essentieel dat er geen data verloren gaat aangezien het gaat om medische data. Het huidige algoritme doet een HTTP call wanneer een waarde wordt ingegeven. Wanneer dit niet lukt, veronderstelt het algoritme dat de applicatie offline is en wordt de data lokaal opgeslagen. Wanneer het toestel terug online komt, dan worden de offline data gesynchroniseerd met de server.
\section{Stand van zaken}
\label{sec:stand-van-zaken}
Wanneer een ontwikkelaar de end user van zijn web -of mobiele applicatie wil voorstellen, dan denkt hij vaak aan een user met dezelfde eigenschappen als zichzelf (laatste smartphone, up-to-date besturingssysteem,  snelle internetverbinding). De ontwikkelomgeving (snelle desktop/laptop met een betrouwbare en snelle internetverbinding) simuleert amper de omgeving van de end user. 
%% TODO: deze sectie (die je kan opsplitsen in verschillende secties) bevat je
\section{Probleemstelling en Onderzoeksvragen}
\label{sec:onderzoeksvragen}

%% TODO:
%% Uit je probleemstelling moet duidelijk zijn dat je onderzoek een meerwaarde
%% heeft voor een concrete doelgroep (bv. een bedrijf).
%%
%% Wees zo concreet mogelijk bij het formuleren van je
%% onderzoeksvra(a)g(en). Een onderzoeksvraag is trouwens iets waar nog
%% niemand op dit moment een antwoord heeft (voor zover je kan nagaan).

\section{Opzet van deze bachelorproef}
\label{sec:opzet-bachelorproef}

%% TODO: Het is gebruikelijk aan het einde van de inleiding een overzicht te
%% geven van de opbouw van de rest van de tekst. Deze sectie bevat al een aanzet
%% die je kan aanvullen/aanpassen in functie van je eigen tekst.

De rest van deze bachelorproef is als volgt opgebouwd:

In Hoofdstuk~\ref{ch:methodologie} wordt de methodologie toegelicht en worden de gebruikte onderzoekstechnieken besproken om een antwoord te kunnen formuleren op de onderzoeksvragen.
%% TODO: Vul hier aan voor je eigen hoofstukken, één of twee zinnen per hoofdstuk
In Hoofdstuk~\ref{ch:glossarium} vindt u de volledige woordenlijst met alle technische begrippen.
In Hoofdstuk~\ref{ch:conclusie}, tenslotte, wordt de conclusie gegeven en een antwoord geformuleerd op de onderzoeksvragen. Daarbij wordt ook een aanzet gegeven voor toekomstig onderzoek binnen dit domein.


%%=============================================================================
%% Methodologie
%%=============================================================================

\chapter{Methodologie}
\label{ch:methodologie}

%% TODO: Hoe ben je te werk gegaan? Verdeel je onderzoek in grote fasen, en
%% licht in elke fase toe welke stappen je gevolgd hebt. Verantwoord waarom je
%% op deze manier te werk gegaan bent. Je moet kunnen aantonen dat je de best
%% mogelijke manier toegepast hebt om een antwoord te vinden op de
%% onderzoeksvraag.

\section{Research}
De eerste stap in het onderzoek was het in kaart brengen van bestaande synchronisatie technologie\"en en technieken voor data synchronisatie. Synchronisatie technologie\"en zoals Firebase, CouchBase en Azure Mobile Apps laten toe om na wat configuratie een applicatie 'offline first' te maken waarbij conflict resolution nagenoeg automatisch verloopt. CouchDB en PouchDB zijn twee andere technologie\"en waarbij er geen 'all-in-one' oplossing wordt aangeboden. Ze werken enkel in tandem voor een oplossing te bieden naar synchronisatie. Wanneer een synchronisatie framework wordt gebruikt, dan gebruikt de client applicatie een specifieke library die de data lokaal beheert indien de gebruiker offline gaat of de client applicatie de data gewoon wil cachen. Wanneer de gebruiker terug online gaat of de client applicatie de gecachte data wil synchroniseren, dan gebeurt de synchronisatie en conflict resolution volledig automatisch. Voorwaarden om dit te kunnen realiseren is de implementatie van een specifieke library om de lokale data te beheren en het gebruik van de correcte backend frameworks of databases.
Door de beperkingen van de business case (.cfr "Beperkingen door business case") was er ook nood aan onderzoek naar synchronisatie patterns. Deze patterns zijn vaak ge\"integreerd in de vermelde synchronisatie technologie\"en.
\subsection{Beperkingen door business case}
Er zijn heel wat oplossingen voor synchronisatie van offline data out-of-the-box maar die hebben bepaalde voorwaarden die niet voldoen aan de constraints van de business case. De use case van Pridiktiv laat immers geen volledige refactor van de backend toe. Men heeft gekozen voor DynamoDB voor de persistentie van de data en helaas biedt DynamoDB geen automatische synchronisatie toe. Hierdoor moeten ook synchronisatie methodes worden onderzocht.
\section{Protoyping}
Om de use case van Pridiktiv te simuleren, wordt er gebruik gemaakt van een prototype dat specifieke scenario's van gebruikers kan simuleren. Net zoals de applicatie van Pridiktiv, is het prototype een Angular applicatie met Ngrx state container. Configuratie van de offline data opslag staat niet vast, zodat het gemakkelijk kan worden aangepast in de testopstelling. Als backend wordt er gebruik gemaakt van een serverless architectuur met AWS Lambda.
\section{Testen van prototype}
Op basis van de research en het prototype, worden enkele test scenario's geselecteerd voor data synchronisatie. Het prototype bestaat uit 2 onderdelen. In de client-side applicatie worden verschillende scenario's overlopen die ook in de business case van Pridiktiv zitten. Het server-side gedeelte bevat net als de business case verschillende microservices die elk een scenario omvatten.
\section{Conclusie}
In het laatste onderdeel van het onderzoek wordt er een synthese gevormd met de resultaten van de testen op het prototype. Die synthese vormt de basis voor de aanbevelingen voor de business case en is het resultaat van het onderzoek naar verschillende synchronisatie opties.


%%=============================================================================
%% Setup
%%=============================================================================

\chapter{Prototype}
\label{ch:prototype}

%% Bespreken setup van prototype

TODO Bespreken van prototype setup
%%=============================================================================
%% Synchronisatie methodes
%%=============================================================================

\chapter{Synchronisatie patterns}
\label{ch:synchronisatiemethdodes}

%% Bespreken synchronisatie patterns

In dit hoofdstuk worden enkele synchronisatie methodes en technieken besproken die data kan synchroniseren met de databank. Indien van toepassing, wordt telkens een concrete situatie van de use case van Pridiktiv gebruikt ter illustratie. Op het einde van dit hoofdstuk worden ook nog andere technieken besproken die niet meteen kunnen worden onderverdeeld onder een specifiek pattern.

\section{Read-Only Data}
Dit data synchronisatie patroon wordt gebruikt wanneer de end user enkel maar data moet kunnen opvragen terwijl de applicatie offline is en die data niet hoeft te manipuleren of de manipulaties op die data niet belangrijk genoeg zijn om te persisteren. Bij Read-Only Data is de richting van het dataverkeer unidirectioneel, van server naar end-user applicatie. Het Read-Only pattern hanteert volgende logica:
\begin{enumerate}
\item de end-user (client) vraagt de data op van de server. De server is de SSOT en houdt alle data bij. Die data kan wijzigen wanneer de end-user bijvoorbeeld offline is
\item server retourneert de data. De opgevraagde data wordt lokaal opgeslagen
\item alle manipulaties op de data worden geblokkeerd en worden nooit doorgegeven aan de server. De client kan dus enkel maar GET HTTP requests sturen voor de data op te vragen
\item bij synchronisatie wordt de oude data lokaal verwijderd en vervangen door de nieuwe data
\item de applicatie controleert op regelmatige tijdstippen de data
\end{enumerate}
Een voorbeeld van het Read-Only pattern van in de use case van Pridiktiv is het opvragen van de pati\"entenlijst. Die lijst kan worden gewijzigd door de hoofdverpleger in het dashboard maar niet in de applicatie zelf. Wanneer de applicatie offline is, kan eenvoudig worden verder gewerkt met de applicatie. Indien er een nieuwe pati\"ent wordt toegevoegd, dan is die vanaf de volgende synchronisatie zichtbaar.
\section{Read-Only Data Optimized}
Het Read-Only Data Optimized pattern is identiek aan het Read-Only Data pattern maar met 1 verschil. Bij de synchronisatie worden enkel de data opgevraagd die gewijzigd zijn en niet alle data. Dit kan op verschillende manieren worden ge\"implementeerd. Er kan een timestamp worden bijgehouden van de laatste wijziging of een version number die incrementeert bij elke wijziging. Op basis van de vergelijking tussen de bestaande data en de data in de databank, kan de applicatie al dan niet beslissen om de lokale data te updaten. De richting van het dataverkeer is bidirectioneel, wat ook verschilt met de standaard implementatie van het Read-Only pattern. Omdat de server eerst moet worden gevraagd of er al dan niet data zijn gewijzigd, moet de applicatie ook kunnen communieren met de server.
\section{Read/Write Data Last Write Wins}
In dit pattern gaat de server er van uit dat writes altijd in de juiste volgorde worden uitgevoerd en de laatste write die naar de databank wordt gestuurd ook de werkelijke laatste wijziging is van de client applicatie. Er wordt geen conflict resolution uitgevoerd. Het pattern blinkt uit wanneer er enkel wordt toegevoegd en de data nooit wordt gemanipuleerd. In de use case van Pridiktiv kan dit worden toegepast bij bijvoorbeeld het toevoegen van notities bij een patient. De notities van een patient worden niet gewijzigd waardoor er geen data kunnen worden overschreven.
\section{Read/Write with Conflict Detection}
Het Read/Write with Conflict Detection pattern is het meest complexe waarbij verschillende end-users dezelfde data wijzigen op het moment dat de applicatie offline is. Bij dit pattern wordt er gesproken van multi-way synchronisatie waarbij een applicatie zowel de data van de server kan updaten en dat de server op zijn beurt alle andere apparaten moet updaten. Een mogelijke flow van het proces zou er als volgt kunnen uitzien:
\begin{enumerate}
\item de server database houdt alle data bij
\item de applicatie houdt lokaal een subset van de data bij die kan worden gewijzigd
\item bij synchronisatie worden de aangepaste data die lokaal wordt opgeslagen naar de server en omgekeerd
\item in de server worden de data aangepast en alle conflicting changes worden geregistreerd voor verdere behandeling
\item Vraagt de gebruiker voor conflict resolution of de applicatie kan zelf het conflict oplossen en de data aanpassen
\end{enumerate}
Een voorbeeld uit de use case van Pridiktiv die gebruik zou kunnen maken van het Read/Write with Conflict Resolution pattern is de opvolging bij wondzorg. Wanneer een end-user een bepaalde wijziging aanbrengt in het dossier van de patient, is het belangrijk dat die data niet verloren gaan indien een andere end-user ook het wondzorg dossier van een patient aanpast. Het is belangrijk om conflict resolution 'achter de schermen' op te lossen om op die manier ervoor te zorgen dat er geen aanpassingen verloren gaan.

%%=============================================================================
%% Onderzoek
%%=============================================================================

\chapter{Onderzoek}
\label{ch:onderzoek}

In dit hoofdstuk wordt dieper ingegaan op de verschillende methodes  en de real-world toepassing van die methodes met de testopstelling. Zoals eerder reeds aangegeven is, worden komen volgende methodes of problemen aan bod:
\begin{itemize}
\item Read-only Optimised
\item Last/First Write Wins
\item Conflict resolution
\end{itemize}

\section{Read-Only Optimised}
De belangrijkste insteek voor deze methode is om zo effici\"ent mogelijk gebruik te maken van de bandbreedte van de client. Deze techniek laat toe om enkel nieuwe data binnen te halen waardoor data die de client reeds bezit niet opnieuw worden opgevraagd. De complexiteit van deze methode stijgt evenredig met het aantal parameters dat in de data kan worden aangepast. Van alle methodes die in dit hoofdstuk worden besproken, is dit de enige methode waarbij enkel een HTTP GET wordt gebruikt.
\subsection{Overzicht: Client}
Bij Read-Only Optimised staat de synchronisatie van de client centraal. Tijdens de periode dat de client offline was, is het mogelijk dat er nieuwe data zijn aangemaakt. Om deze data te synchroniseren kan de client alle data opnieuw opvragen maar deze manier heeft enkele nadelen. Indien de data klein is zoals enkel tekst, dan is Read-Only Optimised triviaal en kan alle data opnieuw worden opgevraagd. Wanneer er echter grotere data moet worden ingeladen zoals afbeeldingen, dan kan Read-Only Optimised heel wat bandbreedte uitsparen. Dankzij de Ngrx store is het ook gemakkelijk om manipulaties uit te voeren op de data, waardoor er eenvoudig objecten kunnen worden toegevoegd aan de data in de store. Belangrijke voorwaarde voor het uitvoeren van deze actie is het bijhouden in de client van de timestamp van de laatste update indien de data aanwezig is. Indien er geen timestamp aanwezig is, dan weet de client dat alle data moet worden opgevraagd.

< TOEVOEGEN SCREENSHOT VAN RESPONSE  BODY ZONDER READ-ONLY OPTIMISED>
< TOEVOEGEN SCREENSHOT VAN RESPONSE  BODY MET READ-ONLY OPTIMISED>

Bovenstaande afbeelding is een voorbeeld van een Read-Only Optimised request waarbij enkel de nieuwe data wordt aangevraagd. Wanneer de client een HTTP GET uitvoert, dan wordt in de header van de request de timestamp van de laatste update toegevoegd. Die timestamp is belangrijk server side, voor het uitvoeren van de query op de databank.

< TOEVOEGEN CODE REDUCER OF SERVICE DIE RESPONS MAPPED >
< TOEVOEGEN SCREENSHOT REDUX STORE>

In de bovenstaande reducer functie wordt een nieuw item toegevoegd aan de data store.

<TOEVOEGEN SCREENSHOT USE CASE IN SETUP APPLICATIE >
De wijzigingen zijn dan meteen zichtbaar in de store en bijgevolg ook in de client.
\subsection{Overzicht: Server}
De server heeft de net zoals bij een traditionele GET de verantwoordelijkheid om de data op te vragen uit de databank en die terug te sturen naar de client. Het enige verschil is dat nu rekening moet worden gehouden met de een mogelijke timestamp in de header. Op basis van die timestamp kan dan alle nieuwe data sinds de laatste GET worden opgevraagd. De request wordt doorgestuurd naar 

<TOEVOEGEN SCREENSHOT OVERZICHT BACKEND INFRASTRUCTUUR VOOR READ-ONLY OPTIMISED>

\subsection{Opmerkingen}
Ondanks de perceptie dat dit een eenvoudige methode is, zijn er enkele opmerkingen zij deze methode:
\begin{enumerate}
\item In bovenstaand scenario wordt er enkel maar uitgegaan van nieuwe data, dus CREATE operaties. Indien de methode ook voor UPDATE moet werken is het in bepaalde gevallen noodzakelijk om ook de databank structuur aan te passen.
\item Bij klassieke SQL databanken kan gemakkelijk worden gefilterd op bepaalde velden zoals een 'updated veld' binnen een tabel. Bij DynamoDB is dit moeilijker als de waarde van de query geen deel uitmaakt van de sort -of partition key. In dat geval moet er dan gebruik worden gemaakt van een extra secondary index zodat er geen full table scan moet worden uitgevoerd.
\item Zoals reeds aangegeven, is het bij kleine data vaak eenvoudiger om steeds alle data op te vragen en geen gebruik te maken van de Read-Only Optimised methode.
\end{enumerate}

\section{Last/First Write Wins}
Wanneer er potenti\"ele conflicten kunnen optreden.



%% Voeg hier je eigen hoofdstukken toe die de ``corpus'' van je bachelorproef
%% vormen. De structuur en titels hangen af van je eigen onderzoek. Je kan bv.
%% elke fase in je onderzoek in een apart hoofdstuk bespreken.

%%=============================================================================
%% Conclusie
%%=============================================================================

\chapter{Conclusie}
\label{ch:conclusie}

%% TODO: Trek een duidelijke conclusie, in de vorm van een antwoord op de
%% onderzoeksvra(a)g(en). Wat was jouw bijdrage aan het onderzoeksdomein en
%% hoe biedt dit meerwaarde aan het vakgebied/doelgroep? Reflecteer kritisch
%% over het resultaat. Had je deze uitkomst verwacht? Zijn er zaken die nog
%% niet duidelijk zijn? Heeft het ondezoek geleid tot nieuwe vragen die
%% uitnodigen tot verder onderzoek?

Afhankelijk van de restricties waarbinnen een applicatie moet worden gebouwd, kan men voor online/offline synchronisatie gebruik maken van een volledig oplossing zoals Microsoft Azure Applications, CouchDB en Google's Firebase. Deze opties werden maar kort toegelicht en de synthese is hierbij dat de synchronisatie automatisch verloopt door middel van database replicatie. Een allesomvattende oplossing is echter niet geschikt voor elke use case. In het voorbeeld van Pridiktiv is de wens er om een van de fully-managed database te gebruiken die Amazon Web Services aanbiedt. Het hanteren van een serverless architectuur met AWS Lambda's is een tweede belangrijke reden om zelf de synchronisatie oplossing te ontwikkelen voor de business applicaties.

Wanneer online/offline synchronisatie moet worden ge\"implementeerd in een applicatie,  bestaat de eerste stap om de use cases onder te verdelen in de verschillende manieren voor synchronisatie. Wanneer absoluut geen data mag verloren gaan moet men resoluut kiezen voor conflict resolution. Wanneer niet alle data belangrijk is, kan er worden geopteerd voor een First/Last Write Wins techniek waarbij er onherroepelijk informatie verloren gaat. De belangrijkste conclusie die men uit het onderzoek kan trekken is dat er geen allesomvattende methode bestaat maar eerder een verzameling aan technieken op om synchronisatie van data uit de client en server te waarborgen. Wanneer er caching en synchronisatie wordt ingebouwd is het belangrijk om ook rekening te houden met performantie en effici\"entie bij het verdelen van de verantwoordelijkheid tussen client -en server side. Voor de applicatie van Pridiktiv, waarbij er met gevoelige medische data wordt gewerkt, is het absoluut noodzakelijk dat er geen belangrijke medische data verloren gaat en er zoveel mogelijk gesynchroniseerd word met de achterliggende backend. Ook naar user experience is synchronisatie een belangrijke factor. Eindgebruikers hebben enkel maar baat bij een robuste offline en online synchronisatie. 
\clearpage
De applicatie kan bij een onderbreking in de verbinding nog verder worden gebruikt en bij het online gaan wordt alle gecachte data automatisch gesynchroniseerd met de server. Teruggekoppeld naar de use case van Pridiktiv, houdt dit in dat de applicatie ook zonder problemen kan worden gebruikt op locaties zoals woonzorgcentra waarbij er vaak maar beperkte of geen internetverbinding beschikbaar is.

Samengevat kan er worden gesteld dat het probleem van synchronisatie zeer specifiek is aan de use case en de data  die men wenst te persisteren. Een interessante piste voor Amazon is dan ook zonder twijfel een service waarbij data uit mobiele applicaties of andere externe data bronnen gesynchroniseerd wordt met DynamoDB (of Aurora, een andere managed databank die Amazon Web Services aanbiedt). Op die manier zou het dan ook mogelijk zijn om bij complexe use cases de data te synchroniseren met de databank. Dit zou zonder twijfel de aantrekkelijkheid van AWS verhogen. Terwijl het onderzoek zich voornamelijk heeft gefocust op data synchronisatie en caching, kan het interessant zijn om bijvoorbeeld de kost in kaart te brengen om over stappen van een fully-managed database naar een andere data source provider die synchronisatie toelaat zonder dat men zelf veel rekening moet houden met synchronisatie.

%%---------- Back matter ------------------------------------------------------

\printbibliography
\addcontentsline{toc}{chapter}{\textcolor{maincolor}{\IfLanguageName{dutch}{Bibliografie}{Bibliography}}}

\listoffigures
\listoftables

\end{document}
