%%=============================================================================
%% Glossarium
%%=============================================================================

\chapter{Glossarium}
\label{ch:glossarium}

%% TODO: Woordenlijst met alle begrippen

\begin{description}
\item[Single-Page Application, SPA] \hfill \ Een single-page application (SPA) is een web applicatie of website waarbij noodzakelijke HTML, CSS en JavaScript dynamisch wordt ingeladen of bij de eerste page load, afhankelijk van de acties van de gebruiker. De volledige pagina wordt bij een SPA nooit volledig herladen. Het is wel mogelijk dat onderdelen van de pagina dynamisch wordt gewijzigd. De data van de SPA wordt dynamisch opgevraagd van de web server. 
\item[Single Source Of Truth, SSOT] \hfill \ In de context van het ontwerp van informatica systemen is de single source of truth een techniek of informatie en data structuren op een bepaalde manier te structureren zodat die niet gedupliceerd wordt. Alle informatie wordt opgehaald in een centraal punt en dat is voor de applicatie en haar componenten de enige plaats waar die informatie kan worden opgehaald. Ngrx store vervult die rol in een Angular applicatie.
\item[Angular CLI] \hfill \ Angular CLI is een command line interface tool waarbij een volledige Angular project wordt gebouwd met minimale configuratie. Er wordt een basis structuur, tests, root module en root component aangemaakt. Met behulp van Webpack worden alle files dan gebundeld in enkele static files. Dankzij Angular CLI wordt er heel wat tijd uitgespaard omdat een groot deel van de configuratie wegvalt. Angular CLI wordt gebruikt de business case van Pridiktiv en in het prototype.
\item[localForage] \hfill \ localForage is een library van Mozilla waarbij het mogelijk om data lokaal te cachen in asynchrone storage mogelijkheden als IndexedDB en WebSQL met een localStorage-achtige syntax. Dankzij deze library is het gemakkelijk om om verschillende soorten data (waaronder ook BLOBS) te cachen. De localStorage DOM Storage wordt wel gebruikt indien de webbrowser geen ondersteuning biedt voor IndexedDB of WebSQL. In dit geval is de opslagruimte om te cachen wel beperkt.
\item[Progressive enchancement] \hfill \ Progressive enhancement is een ontwikkel strategie bij web development waarbij de focus wordt gelegd op de belangrijkste business requirements die de web applicatie of website moet invullen. Afhankelijk van de end-user's browser en connectie, kunnen er 'lagen' van functionaliteit worden toegevoegd aan de applicatie of website. Op die manier kan er altijd een minimum worden aangeboden aan de end-user.
\item[Conflict resolution] \hfill \ Een applicatie kan data van een remote databank lokaal opslaan voor offline functionaliteit aan te bieden aan de gebruiker. Wanneer de applicatie terug online gaat en de applicatie of remote databank een verschil opmerkt, dan is er sprake van een conflict. Conflict resolution duidt op de methode(s) die worden gebruikt voor het synchroniseren van de data tussen de verschillende databanken.
\item[Asynchroon] \hfill \ Asynchroon of 'Asynchronous' in de context van web development wil zeggen dat een bepaalde handeling niet in real-time is maar periodiek van aard is. Dit is belangrijk wanneer gebruikers geen stabiele verbinding hebben tot een netwerk en wanneer men optimaal gebruik wenst te maken van de beschikbare bandbreedte. Bijvoorbeeld RxJs dat gebaseerd is op de ReactiveX library, is volledig gebouwd rond asynchroon programmeren. 
\end{description}