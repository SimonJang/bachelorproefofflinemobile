%%=============================================================================
%% Glossarium
%%=============================================================================

\chapter{Glossarium}
\label{ch:glossarium}

%% TODO: Woordenlijst met alle begrippen

\begin{description}
\item[Single-Page Application, SPA]: Een single-page application (SPA) is een web applicatie of website waarbij noodzakelijke HTML, CSS en JavaScript dynamisch wordt ingeladen of bij de eerste page load. De volledige pagina wordt bij een SPA nooit volledig herladen. Het is wel mogelijk dat onderdelen van de pagina dynamisch wordt gewijzigd. De data van de SPA wordt dynamisch opgevraagd van de web server. Met SPA is mogelijk een om effici\"ent om te springen met de beschikbare bandbreedte.
\item[Angular CLI]: Angular CLI is een command line interface tool waarbij een volledige Angular project wordt gebouwd met minimale configuratie. Er wordt een basis structuur, tests, root module en root component aangemaakt. Met behulp van Webpack worden alle files dan gebundeld in enkele static files. Met Angular CLI wordt er heel wat tijd uitgespaard omdat een groot deel van de configuratie wegvalt. Angular CLI wordt gebruikt de business case van Pridiktiv.
\item[Progressive enhancement]: Progressive enhancement is een ontwikkelstrategie bij web development waarbij de focus wordt gelegd op de belangrijkste business requirements die de web applicatie of website moet invullen. Afhankelijk van de browser en de connectie van de eindgebruiker, kunnen er 'lagen' van functionaliteit (features) worden toegevoegd aan de applicatie of website. Met deze strategie kan een website of web applicatie zich aanpassen aan de eindgebruiker en altijd een basisfunctionaliteit garanderen.
\item[Single Source Of Truth, SSOT]: In de context van het ontwerp van informatica systemen is de single source of truth een techniek om data op een bepaalde manier te structureren zodat die niet gedupliceerd wordt binnen de applicatie en alle referenties verwijzen naar dezelfde 'bron'. Alle informatie wordt opgehaald in een centraal punt en dat is voor de applicatie en haar componenten de enige plaats waar die informatie kan worden opgehaald. De ngrx/redux store vervult die rol in een Angular applicatie.
\item[Conflict resolution]: Een applicatie kan data van een remote databank lokaal opslaan voor offline functionaliteit aan te bieden aan de gebruiker. Wanneer de applicatie terug online gaat en de applicatie of remote databank een verschil opmerkt, dan is er sprake van een conflict. Conflict resolution duidt op de methode(s) die worden gebruikt voor het synchroniseren van de data tussen de verschillende databanken.
\item[Serverless computing]: Een andere naam voor Function as a Service (FaaS) is een cloud-computing executie model waarbij de cloud provider het opstarten en stoppen van een functie volledig zelf beheert. Het opstarten van een functie is op basis van een event. een mogelijk event is bijvoorbeeld een API call naar een HTTP endpoint van een functie. Met serverless computing hoeft een developer zich niet bezig te houden met het configureren en beheren van verschillende servers of virtual machines. Daarnaast spelen schaalbaarheid en kosteneffici\"entie ook een belangrijke rol bij serverless computing. Als FaaS-gebruiker betaal je enkel voor de executietijd en wanneer er meer rekencapaciteit nodig is, gebeurt de schaling volledig automatisch.
\item[ReactiveX]: Reactive Extensions is een verzameling van operatoren die imperatieve\footnote{Imperatieve programmeertalen kunnen de state van een applicatie veranderen. JavaScript is een voorbeeld van een imperatieve programmeertaal} programmeertalen toelaten om een datasequentie te verwerken zonder rekening te houden met het (a)synchrone karakter van de data.
\item[Event-driven Architectuur]: Een software architectuur waarbij de verschillende componenten events genereren, detecteren en op een gepaste manier reageren. De event-driven architectuur vormt de basis van ReactiveX en serverless computing.
\item[Asynchroon]: Asynchroon of 'Asynchronous' in de context van web development wil zeggen dat een bepaalde handeling niet real-time maar periodiek van aard is. Dit is belangrijk wanneer gebruikers geen stabiele verbinding hebben tot een netwerk, wanneer men optimaal gebruik wenst te maken van de beschikbare bandbreedte of eenvoudigweg bij het uitvoeren van een HTTP request. Een voorbeeld is RxJS dat gebaseerd is op de ReactiveX library en volledig gebouwd rond asynchroon programmeren.
\item[Polling]: Bij polling (of pulling) controleert een applicatie met een vast interval of er op een invoer -of uitvoerapparaat nieuwe data beschikbaar is. Het interval of frequentie wanneer de applicatie polling uitvoert, noemt de pollfrequentie. Polling wordt steeds ge\"initialiseerd door de ontvanger van de informatie. Te frequente polling kan een negatieve impact hebben op de performantie van de databron.
\item[Pushing]: Bij push technologie wordt de request voor een datatransactie ge\"initialiseerd door de server of 'publisher' van de informatie via internet.
\item[Microservices Architecture]: Microservices is een architectuur voor het opstellen van server-side enterprise applicaties. Daarbij worden alle onderdelen opgebouwd als een set van losgekoppelde en samenwerkende services. Elke service heeft een beperkte functionaliteit en een beperkte verantwoordelijkheid die kan worden aangeroepen door andere microservices. Dankzij microservices is het mogelijk om om de functionaliteit van een monolitische backend op te delen in verschillende kleine services.
\end{description}