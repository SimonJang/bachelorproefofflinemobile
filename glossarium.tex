%%=============================================================================
%% Glossarium
%%=============================================================================

\chapter{Glossarium}
\label{ch:glossarium}

%% TODO: Woordenlijst met alle begrippen

\begin{description}
\item[Single-Page Application, SPA]: Een single-page application (SPA) is een web applicatie of website waarbij noodzakelijke HTML, CSS en JavaScript dynamisch wordt ingeladen of bij de eerste page load, afhankelijk van de acties van de gebruiker. De volledige pagina wordt bij een SPA nooit volledig herladen. Het is wel mogelijk dat onderdelen van de pagina dynamisch wordt gewijzigd. De data van de SPA wordt normaal dynamisch opgevraagd van de web server. 
\item[Single Source Of Truth, SSOT]: In de context van het ontwerp van informatica systemen is de single source of truth een techniek om data op een bepaalde manier te structureren zodat die niet gedupliceerd wordt binnen de applicatie en alle referenties verwijzen naar dezelfde 'bron'. Alle informatie wordt opgehaald in een centraal punt en dat is voor de applicatie en haar componenten de enige plaats waar die informatie kan worden opgehaald. Ngrx store vervult die rol in een Angular applicatie.
\item[Angular CLI]: Angular CLI is een command line interface tool waarbij een volledige Angular project wordt gebouwd met minimale configuratie. Er wordt een basis structuur, tests, root module en root component aangemaakt. Met behulp van Webpack worden alle files dan gebundeld in enkele static files. Dankzij Angular CLI wordt er heel wat tijd uitgespaard omdat een groot deel van de configuratie wegvalt. Angular CLI wordt gebruikt de business case van Pridiktiv en in het prototype.
\item[localForage]: localForage is een library van Mozilla waarbij het mogelijk is om data lokaal te cachen. De library maakt gebruik van asynchrone storage mogelijkheden als IndexedDB en WebSQL met een localStorage-achtige syntax. De localStorage DOM Storage wordt wel gebruikt indien de webbrowser geen ondersteuning biedt voor IndexedDB of WebSQL. In dit geval is de opslagruimte om te cachen wel beperkter.
\item[Progressive enchancement]: Progressive enhancement is een ontwikkel strategie bij web development waarbij de focus wordt gelegd op de belangrijkste business requirements die de web applicatie of website moet invullen. Afhankelijk van de end-user's browser en connectie, kunnen er 'lagen' van functionaliteit (features) worden toegevoegd aan de applicatie of website. Op die manier kan er altijd een minimum worden aangeboden aan de end-user.
\item[Conflict resolution]: Een applicatie kan data van een remote databank lokaal opslaan voor offline functionaliteit aan te bieden aan de gebruiker. Wanneer de applicatie terug online gaat en de applicatie of remote databank een verschil opmerkt, dan is er sprake van een conflict. Conflict resolution duidt op de methode(s) die worden gebruikt voor het synchroniseren van de data tussen de verschillende databanken.
\item[Event-driven Architectuur]: Een software architectuur waarbij de verschillende componenten events genereren, detecteren en op een gepaste manier reageren. De event-driven architectuur vormt de basis van ReactiveX en serverless computing.
\item[Asynchroon]: Asynchroon of 'Asynchronous' in de context van web development wil zeggen dat een bepaalde handeling niet real-time maar periodiek van aard is. Dit is belangrijk wanneer gebruikers geen stabiele verbinding hebben tot een netwerk, wanneer men optimaal gebruik wenst te maken van de beschikbare bandbreedte of eenvoudigweg bij het uitvoeren van een HTTP request. Een voorbeeld is RxJs dat gebaseerd is op de ReactiveX library en volledig gebouwd rond asynchroon programmeren.
\item[Polling]: Bij polling (of pulling) wordt op regelmatige intervallen de status van applicatie, databank of datastructuur opgevraagd. Bij polling vraagt de ontvanger voor nieuwe data of updates.
\item[Pushing]: Bij push technologie wordt de request voor een transactie ge\"initialiseerd door de server.
\item[Microservices Architecture]: Microservices is een architectuur voor het opstellen van server-side enterprise applicaties. Daarbij worden alle onderdelen opgebouwd als een set van losgekoppelde en samenwerkende services. Elke service heeft een beperkte functionaliteit die kan worden aangeroepen door andere microservices. Wanneer een microservice dan data nodig heeft die het zelf niet beschikt, dan kan wordt er een andere microservice aangeroepen met een andere verantwoordelijkheid.
\item[Serverless computing]: Een andere naam voor Function as a Service (FaaS) is een cloud-computing executie model waarbij de cloud provider het opstarten en stoppen van een functie volledig zelf beheert. Het opstarten van een functie is op basis van een event. een mogelijk event is bijvoorbeeld een API call naar een HTTP endpoint van een functie. Met serverless computing hoeft een developer zich niet bezig te houden met het configureren en beheren van verschillende virtual machines.
\end{description}