%%=============================================================================
%% Conclusie
%%=============================================================================

\chapter{Conclusie}
\label{ch:conclusie}

%% TODO: Trek een duidelijke conclusie, in de vorm van een antwoord op de
%% onderzoeksvra(a)g(en). Wat was jouw bijdrage aan het onderzoeksdomein en
%% hoe biedt dit meerwaarde aan het vakgebied/doelgroep? Reflecteer kritisch
%% over het resultaat. Had je deze uitkomst verwacht? Zijn er zaken die nog
%% niet duidelijk zijn? Heeft het ondezoek geleid tot nieuwe vragen die
%% uitnodigen tot verder onderzoek?

Afhankelijk van de restricties waarbinnen een applicatie moet worden gebouwd, kan men voor online/offline synchronisatie gebruik maken van de een volledig oplossing waarbij synchronisatie wordt aangeboden als een deel van totaal pakket zoals Microsoft Azure Applications of CouchDB. Deze opties werden maar kort toegelicht en het belangrijkste gegeven is hierbij dat de synchronisatie bijna volledig automatisch verloopt met enkel wat. Een allesomvattende oplossing is echter niet geschikt voor elke use case. In het voorbeeld van Pridiktiv is de nood aan een managed database en het gebruik van een serverless arcitectuur met AWS Lambda's de belangrijkste redenen om zelf de synchronisatie methodes in te bouwen in de huidige applicatie.

Wanneer online/offline synchronisatie moet worden ge\"implementeerd in een applicatie,  bestaat de eerste stap om de use cases onder te verdelen in de verschillende manieren voor synchronisatie. Wanneer absoluut geen data mag verloren gaan moet men resoluut kiezen voor conflict resolution. Wanneer niet alle data belangrijk is, kan er worden geopteerd voor een First/Last Write Wins techniek waarbij er onherroepelijk informatie verloren gaat. De belangrijkste conclusie die men uit het onderzoek kan trekken is dat er geen allesomvattende methode bestaat maar eerder een verzameling aan technieken op om synchronisatie van data uit de client en server te waarborgen. Wanneer er  synchronisatie wordt ingebouwd is het belangrijk om ook rekening te houden met performantie en effici\"entie en de verantwoordelijkheid te verdelen tussen client -en server side. Voor de applicatie van Pridiktiv, waarbij er met gevoelige medische data wordt gewerkt, is het absoluut noodzakelijk dat er geen belangrijke data verloren gaat en alles netjes gesynchroniseerd word met de achterliggende backend. Ook naar user experience is synchronisatie een belangrijke factor. Eindgebruikers hebben enkel maar baat bij een robuuste offline/online synchronisatie. De applicatie kan bij een onderbreking in de connectie nog verder worden gebruikt zonder problemen. Afhankelijk van het belang van de data wordt alles ook zorgvuldig gesynchroniseerd indien de gebruiker terug online gaat. Teruggekoppeld naar de use case van Pridiktiv, houdt dit in dat de applicatie ook zonder problemen kan worden gebruikt op locaties zoals woonzorgcentra waarbij er vaak maar beperkte of geen internetverbinding beschikbaar is.

Samengevat kan er worden gesteld dat het probleem van synchronisatie zeer specifiek is aan de use case en welke data men wenst te persisteren maar wel essentieel is voor een aangename user experience. Een interessante piste voor Amazon is dan ook zonder twijfel een service waarbij data uit mobiele applicaties of andere externe data bronnen gesynchroniseerd wordt met DynamoDB (of Aurora, een andere managed databank die Amazon Web Services aanbiedt). Op die manier zou het dan ook mogelijk zijn om bij complexe use cases de data te synchroniseren met de databank. Dit zou zonder twijfel de aantrekkelijkheid van AWS verhogen. Terwijl het onderzoek zich voornamelijk heeft gefocust op data synchronisatietechnieken, kan het interessant zijn om bijvoorbeeld de kost in kaart te brengen om over stappen van een fully-managed database naar een andere data source provider die synchronisatie toelaat zonder dat men zelf veel rekening moet houden met synchronisatie.