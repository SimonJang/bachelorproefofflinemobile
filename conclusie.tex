%%=============================================================================
%% Conclusie
%%=============================================================================

\chapter{Conclusie}
\label{ch:conclusie}

%% TODO: Trek een duidelijke conclusie, in de vorm van een antwoord op de
%% onderzoeksvra(a)g(en). Wat was jouw bijdrage aan het onderzoeksdomein en
%% hoe biedt dit meerwaarde aan het vakgebied/doelgroep? Reflecteer kritisch
%% over het resultaat. Had je deze uitkomst verwacht? Zijn er zaken die nog
%% niet duidelijk zijn? Heeft het ondezoek geleid tot nieuwe vragen die
%% uitnodigen tot verder onderzoek?

Afhankelijk van de restricties waarbinnen een applicatie moet worden gebouwd, kan men voor online/offline synchronisatie gebruik maken van de een volledig oplossing waarbij synchronisatie wordt aangeboden als een deel van totaal pakket zoals Microsoft Azure Applications of CouchDB. De synchronisatie gebeurt hier quasi volledig automatisch. Deze opties werden maar kort toegelicht tijdens dit onderzoek. Een allesomvattende oplossing is echter niet geschikt voor elke use case. In het voorbeeld van Pridiktiv is de nood aan een managed database en het gebruik van een serverless arcitectuur met AWS Lambda's de belangrijkste redenen om zelf de synchronisatie methodes in te bouwen in de huidige applicatie.

Wanneer online/offline synchronisatie moet worden ge\"implementeerd in een applicatie,  bestaat de eerste stap om de use cases onder te verdelen in de verschillende manieren voor synchronisatie. Wanneer absoluut geen data mag verloren gaan moet men resoluut kiezen voor conflict resolution. Wanneer niet alle data belangrijk is, kan er worden geopteerd voor een First/Last Write Wins techniek waarbij er onherroepelijk informatie verloren gaat. De belangrijkste conclusie die men uit het onderzoek kan trekken is dat er geen allesomvattende methode bestaat maar eerder een verzameling aan technieken op om synchronisatie van data uit de client en server te waarborgen. Wanneer synchronisatie wordt ingebouwd is het belangrijk om ook rekening te houden met performantie en effici\"entie en om de verantwoordelijkheid te verdelen tussen client -en server side.

Een interessante piste voor Amazon is dan ook zonder twijfel een service waarbij data uit mobiele applicaties of andere externe data sources gesynchroniseerd wordt met DynamoDB (of Aurora, een andere managed databank die Amazon Web Services aanbiedt).