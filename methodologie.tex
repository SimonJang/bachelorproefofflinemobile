%%=============================================================================
%% Methodologie
%%=============================================================================

\chapter{Methodologie}
\label{ch:methodologie}
In dit hoofdstuk wordt de modus operandi overlopen die gehanteerd werd tijdens dit onderzoek.
%% TODO: Hoe ben je te werk gegaan? Verdeel je onderzoek in grote fasen, en
%% licht in elke fase toe welke stappen je gevolgd hebt. Verantwoord waarom je
%% op deze manier te werk gegaan bent. Je moet kunnen aantonen dat je de best
%% mogelijke manier toegepast hebt om een antwoord te vinden op de
%% onderzoeksvraag.
\section{Research}
De eerste fase van het onderzoek bestond uit research naar de gebruikte technologie\"en in de business case en naar bestaande synchronisatie mogelijkheden. Bij het uitvoeren van deze fase, viel het op dat er weinig literatuur over het onderwerp. Hierdoor bestaat het overgrote deel van de bronvermeldingen en referenties uit hyperlinks naar documentatie, blogs en artikelen van experts.
\subsection{Research naar gebruikte technologie\"en in de business case}
In de eerste grote fase van het onderzoek bestond uit het in kaart brengen van de verschillende technologie\"en die gebruikt worden in de business case. Op basis van die resultaten zijn de grenzen bepaald waar binnen dit onderzoek wordt uitgevoerd.
\subsection{Beperkingen in de business case}
\label{subsec:beperkingen-business-case}
De business case van Pridiktiv laat geen drastische aanpassingen aan de backend toe. De technologie stack maakt gebruik van DynamoDB voor het persisteren van de data en helaas biedt DynamoDB geen automatische synchronisatie aan. 
\subsection{Research naar bestaande synchronisatie methodes}
De tweede deel van de research fase bestond uit research naar bestaande synchronisatie mogelijkheden. Op basis van het resultaat van het eerste deel en de beperkingen van de technologie stack van de business case, is er een selectie uitgevoerd op de geschikte technologie\"en. Er zijn heel wat oplossingen voor synchronisatie van offline data out-of-the-box maar die hebben bepaalde voorwaarden die niet voldoen aan de constraints van de business case.
\subsection{Bespreken research met co-promoter}

\section{Ontwerpen en prototyping van architectuur voor synchronisatie oplossing}
\section{Toepassen van synchronisatie oplossing op business case}
Het grootste deel van de synchronisatie vindt echter plaatst in de backend. Daarom dient de client applicatie voornamelijk voor het tonen van gesynchroniseerde data en om caching en pre-caching methodes te onderzoeken. Caching is niet de focus van dit onderzoek maar speelt een niet-onbelangrijke rol bij de transitie van online naar offline en omgekeerd. In het hoofdstuk ~\ref{ch:onderzoek} 'Onderzoek' wordt voornamelijk in detail ingegaan op de caching in de client en de synchronisatie in de server. De toepassing van de synchronisatie van het onderzoek van de synchronisatiemethodes vond plaatst tijdens de stage periode bij Pridiktiv. Met behulp van de literatuurstudie en research in de documentatie van verschillende technologie\"en en APIs, is het mogelijk geweest om een oplossing aan te bieden bij de synchronisatie van de data die offline werd gegenereerd. 
\section{Conclusie}
In het laatste onderdeel van het onderzoek wordt er op basis van de resultaten van het onderzoek een synthese gevormd.
