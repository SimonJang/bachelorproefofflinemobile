%%=============================================================================
%% Methodologie
%%=============================================================================

\chapter{Methodologie}
\label{ch:methodologie}

%% TODO: Hoe ben je te werk gegaan? Verdeel je onderzoek in grote fasen, en
%% licht in elke fase toe welke stappen je gevolgd hebt. Verantwoord waarom je
%% op deze manier te werk gegaan bent. Je moet kunnen aantonen dat je de best
%% mogelijke manier toegepast hebt om een antwoord te vinden op de
%% onderzoeksvraag.

\section{Research}
De eerste stap in het onderzoek was het in kaart brengen van bestaande synchronisatie technologie\"en en technieken voor data synchronisatie. Synchronisatie technologie\"en zoals Firebase, CouchBase en Azure Mobile Apps laten toe om na wat configuratie een applicatie 'offline first' te maken waarbij conflict resolution nagenoeg automatisch verloopt. CouchDB en PouchDB zijn twee andere technologie\"en waarbij er geen 'all-in-one' oplossing wordt aangeboden. Ze werken enkel in tandem voor een oplossing te bieden naar synchronisatie. Wanneer een synchronisatie framework wordt gebruikt, dan gebruikt de client applicatie een specifieke library die de data lokaal beheert indien de gebruiker offline gaat of de applicatie de data wil cachen. Wanneer de gebruiker terug online gaat of de applicatie de gecachte data wil synchroniseren, dan gebeurt de synchronisatie en conflict resolution volledig automatisch. Voorwaarden om dit te kunnen realiseren is de implementatie van een specifieke library om de lokale data te beheren en het gebruik van de correcte backend frameworks of databases.
Door de beperkingen van de business case (.cfr "Beperkingen door business case") was er ook nood aan onderzoek naar synchronisatie patterns. Deze patterns zijn vaak ge\"integreerd in de vermelde synchronisatie technologie\"en.
\clearpage
\subsection{Beperkingen door business case}
Er zijn heel wat oplossingen voor synchronisatie van offline data out-of-the-box maar die hebben bepaalde voorwaarden die niet voldoen aan de constraints van de business case. De use case van Pridiktiv laat immers geen volledige refactor van de backend toe. Men heeft gekozen voor DynamoDB voor de persistentie van de data en helaas biedt DynamoDB geen automatische synchronisatie toe. Door deze beperking ligt de focus van dit onderzoek eerder om synchronisatie methodes te bespreken.
\section{Protoyping}
Voor het onderzoek is er een eenvoudige Angular applicatie ontwikkeld die bepaalde use cases van de business case nabootst. Het grootste deel van de synchronisatie vindt echter plaatst in de backend. Daarom dient de client applicatie voornamelijk voor het tonen van gesynchroniseerde data en om caching en pre-caching methodes te onderzoeken. Caching is niet de focus van dit onderzoek maar speelt een niet-onbelangrijke rol bij de transitie van online naar offline en omgekeerd. In het hoofdstuk 'Onderzoek' wordt voornamelijk in detail ingegaan op de interactie van de client en server.
\section{Onderzoeken synchronisatiemethodes}
Het grootste deel van het onderzoek van de synchronisatiemethodes vond plaatst tijdens de stage periode bij Pridiktiv. Met behulp van de literatuurstudie en research in de documentatie van verschillende technologie\"en en APIs, is het mogelijk geweest om een oplossing aan te bieden bij de synchronisatie van de data die offline werd gegenereerd.
\section{Conclusie}
In het laatste onderdeel van het onderzoek wordt er op basis van de resultaten van het onderzoek een synthese gevormd.
