%%=============================================================================
%% Methodologie
%%=============================================================================

\chapter{Methodologie}
\label{ch:methodologie}

%% TODO: Hoe ben je te werk gegaan? Verdeel je onderzoek in grote fasen, en
%% licht in elke fase toe welke stappen je gevolgd hebt. Verantwoord waarom je
%% op deze manier te werk gegaan bent. Je moet kunnen aantonen dat je de best
%% mogelijke manier toegepast hebt om een antwoord te vinden op de
%% onderzoeksvraag.

\section{Research}
De eerste stap in het onderzoek was het in kaart brengen van bestaande synchronisatie technologie\"en en technieken voor data synchronisatie. Synchronisatie technologie\"en zoals Firebase, CouchBase en Azure Mobile Apps laten toe om na wat configuratie een applicatie 'offline first' te maken waarbij conflict resolution nagenoeg automatisch verloopt. CouchDB en PouchDB zijn twee andere technologie\"en waarbij er geen 'all-in-one' oplossing wordt aangeboden. Ze werken enkel in tandem voor een oplossing te bieden naar synchronisatie. Wanneer een synchronisatie framework wordt gebruikt, dan gebruikt de client applicatie een specifieke library die de data lokaal beheert indien de gebruiker offline gaat of de client applicatie de data gewoon wil cachen. Wanneer de gebruiker terug online gaat of de client applicatie de gecachte data wil synchroniseren, dan gebeurt de synchronisatie en conflict resolution volledig automatisch. Voorwaarden om dit te kunnen realiseren is de implementatie van een specifieke library om de lokale data te beheren en het gebruik van de correcte backend frameworks of databases.
Door de beperkingen van de business case (.cfr "Beperkingen door business case") was er ook nood aan onderzoek naar synchronisatie patterns. Deze patterns zijn vaak ge\"integreerd in de vermelde synchronisatie technologie\"en.
\subsection{Beperkingen door business case}
Er zijn heel wat oplossingen voor synchronisatie van offline data out-of-the-box maar die hebben bepaalde voorwaarden die niet voldoen aan de constraints van de business case. De use case van Pridiktiv laat immers geen volledige refactor van de backend toe. Men heeft gekozen voor DynamoDB voor de persistentie van de data en helaas biedt DynamoDB geen automatische synchronisatie toe. Hierdoor moeten ook synchronisatie methodes worden onderzocht.
\section{Protoyping}
Om de use case van Pridiktiv te simuleren, wordt er gebruik gemaakt van een prototype dat specifieke scenario's van gebruikers kan simuleren. Net zoals de applicatie van Pridiktiv, is het prototype een Angular applicatie met Ngrx state container. Configuratie van de offline data opslag staat niet vast, zodat het gemakkelijk kan worden aangepast in de testopstelling. Als backend wordt er gebruik gemaakt van een serverless architectuur met AWS Lambda.
\section{Testen van prototype}
Op basis van de research en het prototype, worden enkele test scenario's geselecteerd voor data synchronisatie. Het prototype bestaat uit 2 onderdelen. In de client-side applicatie worden verschillende scenario's overlopen die ook in de business case van Pridiktiv zitten. Het server-side gedeelte bevat net als de business case verschillende microservices die elk een scenario omvatten.
\section{Conclusie}
In het laatste onderdeel van het onderzoek wordt er een synthese gevormd met de resultaten van de testen op het prototype. Die synthese vormt de basis voor de aanbevelingen voor de business case en is het resultaat van het onderzoek naar verschillende synchronisatie opties.

