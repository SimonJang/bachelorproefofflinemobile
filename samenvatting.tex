%%=============================================================================
%% Samenvatting
%%=============================================================================

%% TODO: De "abstract" of samenvatting is een kernachtige (~ 1 blz. voor een
%% thesis) synthese van het document.
%%
%% Deze aspecten moeten zeker aan bod komen:
%% - Context: waarom is dit werk belangrijk?
%% - Nood: waarom moest dit onderzocht worden?
%% - Taak: wat heb je precies gedaan?
%% - Object: wat staat in dit document geschreven?
%% - Resultaat: wat was het resultaat?
%% - Conclusie: wat is/zijn de belangrijkste conclusie(s)?
%% - Perspectief: blijven er nog vragen open die in de toekomst nog kunnen
%%    onderzocht worden? Wat is een mogelijk vervolg voor jouw onderzoek?
%%
%% LET OP! Een samenvatting is GEEN voorwoord!

\IfLanguageName{english}{%
\selectlanguage{dutch}
\chapter*{Samenvatting}
\lipsum[1-4]
\selectlanguage{english}
}{}

%%---------- Samenvatting -----------------------------------------------------
%%
%% De samenvatting in de hoofdtaal van het document

\chapter*{\IfLanguageName{dutch}{Samenvatting}{Abstract}}

Mobiele applicatieontwikkeling beschouwt toegang tot internet vaak als vanzelfsprekend wanneer een mobiele applicatie wordt gebruikt. Dit is echter niet het geval in elke (werk)omgeving. Daarom moet de ontwikkelaar ervoor zorgen dat de applicatie de data ook lokaal kan opslaan indien de wijzigingen niet meteen kunnen worden doorgevoerd naar de achterliggende infrastructuur. Dit probleem vormt voor ontwikkelaars een uitdaging omdat de data integriteit moet worden gewaarborgd wanneer het toestel terug verbonden is met internet. Het gebruik van een performante en betrouwbare methode voor de data die offline wordt ingegeven te synchroniseren met de online cloud-based databank is dus essentieel. Hierbij is het belangrijk om een onderscheid te maken tussen use cases waarbij de developer gebruik maakt van fully-managed databases zoals onder andere DynamoDB van Amazon Web Services en use cases waarbij men zelf de database beheert op verschillende virtual machines. In het onderzoek is synchronisatie specifiek onderzocht voor de fully-managed database DynamoDB die gebruikt van een serverless microservices architectuur en een Angular applicatie als client applicatie. 

Na onderzoek is gebleken dat er verschillende manieren van synchronisatie zijn.  Wanneer men zelf de synchronisatie wenst te realiseren is het belangrijk om de verschillende use cases onder te verdelen volgens synchronisatie methode en zo verder te werken. Deze methodologie werd ook gehanteerd bij het ontwikkelen van de test applicatie. Er zijn scenario's waar er geen conflicten zijn maar performantie de belangrijkste factor is bv. Read-Only Optimised. Wanneer er conflicten optreden kan men dan kiezen voor Last Write Wins waarbij er onvermijdelijk data verloren gaat of andere strategie\"en van conflict resolution. In het onderzoek komt er een voorbeeld van hoe conflict resolution kan worden opgelost met behulp van timestamps. Met behulp van verschillende conflict resolution technieken is het dus mogelijk om een robuuste applicatie te ontwikkelen die de mogelijkheid heeft om om te gaan met onderbrekingen in de netwerk verbinding.