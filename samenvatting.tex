%%=============================================================================
%% Samenvatting
%%=============================================================================

%% TODO: De "abstract" of samenvatting is een kernachtige (~ 1 blz. voor een
%% thesis) synthese van het document.
%%
%% Deze aspecten moeten zeker aan bod komen:
%% - Context: waarom is dit werk belangrijk?
%% - Nood: waarom moest dit onderzocht worden?
%% - Taak: wat heb je precies gedaan?
%% - Object: wat staat in dit document geschreven?
%% - Resultaat: wat was het resultaat?
%% - Conclusie: wat is/zijn de belangrijkste conclusie(s)?
%% - Perspectief: blijven er nog vragen open die in de toekomst nog kunnen
%%    onderzocht worden? Wat is een mogelijk vervolg voor jouw onderzoek?
%%
%% LET OP! Een samenvatting is GEEN voorwoord!

%%---------- Nederlandse samenvatting -----------------------------------------
%%
%% TODO: Als je je bachelorproef in het Engels schrijft, moet je eerst een
%% Nederlandse samenvatting invoegen. Haal daarvoor onderstaande code uit
%% commentaar.
%% Wie zijn bachelorproef in het Nederlands schrijft, kan dit negeren en heel
%% deze sectie verwijderen.

\IfLanguageName{english}{%
\selectlanguage{dutch}
\chapter*{Samenvatting}
\lipsum[1-4]
\selectlanguage{english}
}{}

%%---------- Samenvatting -----------------------------------------------------
%%
%% De samenvatting in de hoofdtaal van het document

\chapter*{\IfLanguageName{dutch}{Samenvatting}{Abstract}}

Mobiele applicatieontwikkeling beschouwt toegang tot internet vaak als vanzelfsprekend wanneer een mobiele applicatie wordt gebruikt. Dit is echter niet het geval in elke (werk)omgeving. Daarom moet de ontwikkelaar ervoor zorgen dat de applicatie de data ook lokaal kan opslaan indien de wijzigingen niet meteen kunnen worden doorgevoerd naar de achterliggende infrastructuur. Dit probleem vormt voor ontwikkelaars een uitdaging omdat de data integriteit moet worden gewaarborgd wanneer het toestel terug verbonden is met internet. Het gebruik van een performante en betrouwbare methode voor de data die offline wordt ingegeven te synchroniseren met de online cloud-based databank is dus essentieel.

-- TODO verder aanvullen bij voortgang onderzoek
