%%=============================================================================
%% Inleiding
%%=============================================================================
\chapter{Inleiding}
\label{ch:inleiding}
Het onderzoek zal verschillende methodes voor offline opslag en synchronisatie analyseren en onderzoeken. Het onderzoek zal van elke methode de voor- en nadelen overlopen en de toepassing tonen met behulp van een prototype. In de sectie 'Business Case' wordt de business case van Pridiktiv.care - Into.care toegelicht. Daarna worden in 'Terminologie' de verschillende relevante termen overlopen. In 'Stand van zaken' komt de context en noodzaak van het onderzoek aan bod. Tenslotte volgt de probleemstelling, de onderzoeksvraag en de opzet van de bachelorproef.
%% literatuurstudie. Vergeet niet telkens je bronnen te vermelden!
\section{Business Case}
\label{sec:business-case}
De business case van Pridiktiv vormt het vertrekpunt van deze studie en wordt nu nog wat verder toegelicht. De applicatie draait op een smartphone in een woonzorgcentrum. Het woonzorgcentrum beschikt over draadloos internet maar de dekking is niet het volledige gebouw. Hierdoor moet de applicatie ook offline werken. Wanneer verschillende waarden worden geregistreerd met de applicatie (zoals bloeddruk, gewicht, inname medicatie) worden die offline opgeslagen in een IndexedDB. Wanneer het toestel terug online komt, moeten deze waarden worden doorgestuurd naar de backend. Het is essentieel dat er geen data verloren gaat aangezien het gaat om medische data. Het huidige algoritme doet een HTTP call wanneer een waarde wordt ingegeven. Wanneer dit niet lukt, veronderstelt het algoritme dat de applicatie offline is en wordt de data lokaal opgeslagen. Wanneer het toestel terug online komt, dan worden de offline data gesynchroniseerd met de server. De realiteit wijst echter uit dat betrouwbaar internet eerder uitzonderlijk is in een woonzorgcentrum. In deze context is er dus nood aan een 'offline first' -  oplossing. De huidige applicatie is een Angular (het vroegere Angular 2) met ngrx als state container als een Cordova applicatie. Voor lokale opslag wordt momenteel gebruik gemaakt van Mozilla's LocalForage library.
\section{Terminologie}
\label{sec:terminologie}
In deze sectie komen de verschillende componenten van het onderzoek, het prototype en tools aan bod. De volledige lijst met begrippen en termen kan u terugvinden onder het hoofdstuk 'Glossarium'. Er is een minimum kennis in verband met software ontwikkeling voor het web vereist van de lezer om deze sectie volledig te begrijpen.
\subsection{Huidige Web APIs voor lokale opslag}
Er zijn momenteel 4 APIs die ondersteund worden door de verschillende browsers. De verschillende libraries en methodes in dit onderzoek maken steeds gebruik van 1 (of meerdere) beschikbare APIs.
\subsubsection{LocalStorage en SessionStorage}
\subsection{Scalable Single Page Application}
\subsection{ngrx store: Redux - achtige state container}
\subsection{Reactive programming paradigm}
\subsection{Offline First!}
\section{Stand van zaken}
\label{sec:stand-van-zaken}
Wanneer een ontwikkelaar de end user van zijn web -of mobiele applicatie wil voorstellen, dan denkt hij vaak aan een user met dezelfde eigenschappen als zichzelf (laatste smartphone, up-to-date besturingssysteem,  snelle internet verbinding). De ontwikkelomgeving (snelle desktop/laptop met een betrouwbare en snelle internet verbinding) simuleert amper de omgeving van de end user. Ziet de realiteit er van de end user anders uit. Bepaalde omgevingsfactoren zoals bv. tunnels, trein en vliegtuig kunnen hebben een grote invloed op de betrouwbaarheid van de connectie. Het klassieke client - server model waarbij de client enkel maar als het ware een view is van de data die door de server wordt bijgehouden, is achterhaald, want elke onderbreking in de internet connectie zorgt ervoor dat de applicatie niet meer kan worden gebruikt. Dat is een scenario dat ten alle kosten moet worden vermeden.
De focus naar offline functionaliteit wint de laatste jaren aan populariteit nu globaal, snel en betrouwbaar internet nog wat toekomstmuziek blijkt te zijn. Design filosofi\"en zoals 'Offline first' zoals reeds vermeld proberen ontwikkelaars er van te overtuigen om meer aandacht te besteden aan het connectiviteit aspect van hun applicatie.
%% TODO: deze sectie (die je kan opsplitsen in verschillende secties) bevat je
\section{Probleemstelling en Onderzoeksvragen}
\label{sec:onderzoeksvragen}

%% TODO:
%% Uit je probleemstelling moet duidelijk zijn dat je onderzoek een meerwaarde
%% heeft voor een concrete doelgroep (bv. een bedrijf).
%%
%% Wees zo concreet mogelijk bij het formuleren van je
%% onderzoeksvra(a)g(en). Een onderzoeksvraag is trouwens iets waar nog
%% niemand op dit moment een antwoord heeft (voor zover je kan nagaan).

\section{Opzet van deze bachelorproef}
\label{sec:opzet-bachelorproef}

%% TODO: Het is gebruikelijk aan het einde van de inleiding een overzicht te
%% geven van de opbouw van de rest van de tekst. Deze sectie bevat al een aanzet
%% die je kan aanvullen/aanpassen in functie van je eigen tekst.

De rest van deze bachelorproef is als volgt opgebouwd:

In Hoofdstuk~\ref{ch:methodologie} wordt de methodologie toegelicht en worden de gebruikte onderzoekstechnieken besproken om een antwoord te kunnen formuleren op de onderzoeksvragen.
%% TODO: Vul hier aan voor je eigen hoofstukken, één of twee zinnen per hoofdstuk
In Hoofdstuk~\ref{ch:glossarium} vindt u de volledige woordenlijst met alle technische begrippen.
In Hoofdstuk~\ref{ch:conclusie}, tenslotte, wordt de conclusie gegeven en een antwoord geformuleerd op de onderzoeksvragen. Daarbij wordt ook een aanzet gegeven voor toekomstig onderzoek binnen dit domein.

