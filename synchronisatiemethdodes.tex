%%=============================================================================
%% Synchronisatie methodes
%%=============================================================================

\chapter{Synchronisatie methodes}
\label{ch:synchronisatiemethdodes}

%% Bespreken synchronisatie patterns

In dit hoofdstuk worden enkele synchronisatie methodes en technieken besproken die data kunnen synchroniseren met de databank. In sectie~\ref{sec:bestaande} 'Bestaande synchronisatie mogelijkheden' worden kort de bestaande oplossingen overlopen. Alternatieve synchronisatie patronen worden overlopen in sectie~\ref{sec:alternatieven} 'Alternatieve synchronisatie mogelijkheden'.

\section{Bestaande synchronisatie mogelijkheden}
\label{sec:bestaande}
Op basis van die research en de beperkingen van de business case was het immers eenvoudiger om te bepalen welke technologie\"en al dan niet in aanmerking komen voor het onderzoek. Synchronisatie technologie\"en zoals Firebase\autocite{firebase-sync}, CouchDB\autocite{sync-couchdb} en Azure Mobile Apps\autocite{azure-sync} laten eenvoudig toe dat een applicatie de data synchroniseert met de backend. De oplossingen maken gebruik van database replicatie. De client kopie\"ert een deel van de database en bij het synchroniseren wordt de database in de backend dan aangepast op basis van de lokale kopie die de client bezit. De database rechtstreeks vanuit de client applicatie manipuleren brengt ook enkele andere nadelen met zich mee. De client applicatie en de databankstructuur zijn hard gekoppeld. Hierdoor is het moeilijk om de database te refactoren dan moet de client ook worden aangepast. De client kan ook bepaalde handelingen op de database uitvoeren die niet wenselijk zijn. Dit maakt bestaande synchronisatie oplossingen niet aantrekkelijk om het probleem van de business case op te lossen.

\section{Alternatieve synchronisatie mogelijkheden}
\label{sec:alternatieven}
Door de beperkingen van de business case is het noodzakelijk om andere methodes te onderzoeken. In het volgende deel worden methodes\autocite{outsystem-sync} overlopen die zowel kunnen gebruikt worden bij caching als bij synchronisatie. In hoofdstuk~\ref{ch:onderzoek} 'Onderzoek' wordt er dieper ingegaan op de toepassing van de verschillende methodes op de business case.

\subsection{Read-Only}
Deze caching methode laat toe dat de gebruiker de applicatie verder kan gebruiken wanneer die offline is. Door het beperken van de interacties in de client naar Read-Only is in dit scenario geen sprake van synchronisatie. De gebruiker mag immers de data niet manipuleren. Bij Read-Only is de richting van het dataverkeer unidirectioneel, van server naar client applicatie. De methode heeft volgende eigenschappen.
\begin{itemize}
\item De client vraagt de data\footnote{Applicatie moet online zijn om de initi\"ele state op te bouwen.} op van de server. De server is de SSOT en houdt alle data bij. Die data kan wijzigen wanneer de client offline is.
\item De server retourneert de data. De opgevraagde data wordt gecached in de store. De store is nu de nieuwe SSOT voor de client.
\item Alle manipulaties op de data worden geblokkeerd wanneer de applicatie offline is. De client dient enkel als view voor de data die lokaal is bijgehouden.
\item De client synchroniseren wanneer die terug online is houdt in dat de oude data lokaal worden overschreven door de nieuwe opgevraagde data.
\item Wanneer een view in de client wordt geladen, probeert de client de server te pollen voor nieuwe data. Indien de applicatie offline is, wordt de gecachte data uit de store gebruikt.
\end{itemize}

\subsection{Read-Only Optimized}
De Read-Only Optimized caching methode is bijna identiek aan de Read-Only methode maar met enkele verschillen. Bij het synchroniseren worden enkel de data opgevraagd die gewijzigd zijn. Dit kan op verschillende manieren worden ge\"implementeerd. Er kan een timestamp worden bijgehouden van de laatste wijziging of een version number die incrementeert bij elke wijziging. Op basis van de vergelijking tussen de lokale data en de data in de databank, kan de applicatie al dan niet beslissen om de lokale data te overschrijven. De richting van het dataverkeer is bidirectioneel omdat de client data moet meesturen met de request. Een nadeel van de Read-Only Optimised caching is de complexiteit die deze methode introduceert.

Read-Only Optimized kan ook server side wordt gebruikt voor het samenstellen van de state\footnote{Redux state in JSON, voor de redux store in de client}. Hier wordt verder op ingegaan in het hoofdstuk~\ref{ch:onderzoek} 'Onderzoek'.

\subsection{First/Last Write Wins}
Bij de First/Last Write Wins techniek gaat de server er van uit dat writes steeds in de juiste volgorde worden uitgevoerd. Er zijn twee varianten die elk, afhankelijk van de use case, de data kunnen synchroniseren. Bij First Write Wins zijn de data impliciet final. De data kunnen dan maar een keer worden aangepast. Alle andere pogingen worden verworpen. Bij de tweede variant Last Write Wins, worden de data steeds overschreven door de meest recente versie en kunnen de data op ieder moment worden aangepast door recentere data. De techniek wordt enkel maar server side gebruikt en verwerkt de gecachte\footnote{De client bewaart de acties wanneer er geen verbinding is met de server. Zie hoofdstuk~\ref{ch:onderzoek}} transacties van de client.

Met deze methode gaat er onherroepelijk data verloren. Bij sommige use cases kan deze compromis worden gemaakt. Indien dit niet mogelijk is dan biedt~\ref{subsec: conflict-detection} 'Write with Conflict Detection' een betere oplossing.

\subsection{Write with Conflict Detection}
\label{subsec: conflict-detection}
Het Read/Write with Conflict Detection methode is het complexere variant van First/Last Write Wins. Bij deze techniek ligt de focus om minimale data te verliezen bij de synchronisatie. Wanneer er server side data moet worden overschreven zoals bij Last Write Wins, dan wordt de bestaande data toegevoegd aan de nieuwe data. Bij First-Write Wins kan er ook data worden toegevoegd zonder dat de oorspronkelijke data wordt overschreven. Indien de server een conflict vaststelt waarbij er geen beslissing kan worden genomen\footnote{Bv. een UPDATE op een record dat verwijderd is door een DELETE operatie}, dan moet de eindgebruiker worden verwittigd. De gebruiker maakt dan de keuze welke data er moeten worden bewaard en welke er mogen worden overschreven. Bij voorkeur lost de server het conflict autonoom op, zonder tussenkomst van de gebruiker. Dit is echter niet bij elke use case mogelijk. 
Write with Conflict Detection zou volgende stappen kunnen doorlopen: 
\begin{enumerate}
\item de server houdt al de data bij
\item de applicatie houdt lokaal een subset van de data bij die kan worden gewijzigd
\item bij synchronisatie worden de aangepaste data die lokaal wordt opgeslagen naar de server
\item in de server worden de data aangepast en alle conflicten worden geregistreerd voor verdere behandeling
\item de server verwittigt de gebruiker voor een conflict of de server kan zelf het conflict oplossen en de data aanpassen
\item de client vraagt de meest recente data terug
\end{enumerate}