%%=============================================================================
%% Synchronisatie methodes
%%=============================================================================

\chapter{Synchronisatie patterns}
\label{ch:synchronisatiemethdodes}

%% Bespreken synchronisatie patterns

In dit hoofdstuk worden enkele synchronisatie methodes en technieken besproken die data kan synchroniseren met de databank. Indien van toepassing, wordt telkens een concrete situatie van de use case van Pridiktiv gebruikt ter illustratie. Op het einde van dit hoofdstuk worden ook nog andere technieken besproken die niet meteen kunnen worden onderverdeeld onder een specifiek pattern.

\subsection{Conclusie research}
Op basis van die research en de beperkingen van de business case was het immers eenvoudiger om te bepalen welke technologie\"en al dan niet in aanmerking komen voor het onderzoek. Synchronisatie technologie\"en zoals Firebase, CouchBase en Azure Mobile Apps laten toe dat een applicatie offline functioneert met conflict resolution. Dit proces verloopt nagenoeg automatisch verloopt aan de hand van synchronisatie policies of door tussenkomst van de gebruiker. Door de manier\footnote{Met databank replicatie, waarbij rechtstreeks op de database wordt gewerkt} hoe bestaande synchronisatie oplossingen zoals Firebase en Azure werken, is het niet mogelijk de bestaande oplossingen toe te passen op de business case. 

CouchDB en PouchDB zijn twee andere technologie\"en waarbij er geen 'all-in-one' oplossing wordt aangeboden. Helaas is PouchDB enkel compatibel met CouchDB en derivaten. Waardoor het niet geschikt is voor deze business case om een oplossing te bieden naar synchronisatie.

Wanneer een synchronisatie framework wordt gebruikt, dan gebruikt de client applicatie een specifieke library die de data lokaal beheert indien de gebruiker offline gaat of de applicatie de data wil cachen. Wanneer de gebruiker terug online gaat of de applicatie de gecachte data wil synchroniseren, dan gebeurt de synchronisatie en conflict resolution volledig automatisch. Voorwaarden om dit te kunnen realiseren is de implementatie van een specifieke library om de lokale data te beheren en het gebruik van de correcte backend frameworks of databases.
Door de beperkingen van de business case ~\ref{subsec:beperkingen-business-case} was er ook nood aan onderzoek naar synchronisatie patterns. Deze patterns zijn vaak ge\"integreerd in de vermelde synchronisatie technologie\"en.

\section{Read-Only Data}
Dit data synchronisatie patroon wordt gebruikt wanneer de end user enkel maar data moet kunnen opvragen terwijl de applicatie offline is en die data niet hoeft te manipuleren of de manipulaties op die data niet belangrijk genoeg zijn om te persisteren. Bij Read-Only Data is de richting van het dataverkeer unidirectioneel, van server naar end-user applicatie. Het Read-Only pattern hanteert volgende logica:
\begin{enumerate}
\item de end-user (client) vraagt de data op van de server. De server is de SSOT en houdt alle data bij. Die data kan wijzigen wanneer de end-user bijvoorbeeld offline is
\item server retourneert de data. De opgevraagde data wordt lokaal opgeslagen
\item alle manipulaties op de data worden geblokkeerd en worden nooit doorgegeven aan de server. De client kan dus enkel maar GET HTTP requests sturen voor de data op te vragen
\item bij synchronisatie wordt de oude data lokaal verwijderd en vervangen door de nieuwe data
\item de applicatie controleert op regelmatige tijdstippen de data
\end{enumerate}
Een voorbeeld van het Read-Only pattern van in de use case van Pridiktiv is het opvragen van de pati\"entenlijst. Die lijst kan worden gewijzigd door de hoofdverpleger in het dashboard maar niet in de applicatie zelf. Wanneer de applicatie offline is, kan eenvoudig worden verder gewerkt met de applicatie. Indien er een nieuwe pati\"ent wordt toegevoegd, dan is die vanaf de volgende synchronisatie zichtbaar.
\section{Read-Only Data Optimized}
Het Read-Only Data Optimized pattern is identiek aan het Read-Only Data pattern maar met 1 verschil. Bij de synchronisatie worden enkel de data opgevraagd die gewijzigd zijn en niet alle data. Dit kan op verschillende manieren worden ge\"implementeerd. Er kan een timestamp worden bijgehouden van de laatste wijziging of een version number die incrementeert bij elke wijziging. Op basis van de vergelijking tussen de bestaande data en de data in de databank, kan de applicatie al dan niet beslissen om de lokale data te updaten. De richting van het dataverkeer is bidirectioneel, wat ook verschilt met de standaard implementatie van het Read-Only pattern. Omdat de server eerst moet worden gevraagd of er al dan niet data zijn gewijzigd, moet de applicatie ook kunnen communieren met de server.
\section{Read/Write Data Last Write Wins}
In dit pattern gaat de server er van uit dat writes altijd in de juiste volgorde worden uitgevoerd en de laatste write die naar de databank wordt gestuurd ook de werkelijke laatste wijziging is van de client applicatie. Er wordt geen conflict resolution uitgevoerd. Het pattern blinkt uit wanneer er enkel wordt toegevoegd en de data nooit wordt gemanipuleerd. In de use case van Pridiktiv kan dit worden toegepast bij bijvoorbeeld het toevoegen van notities bij een patient. De notities van een patient worden niet gewijzigd waardoor er geen data kunnen worden overschreven.
\section{Read/Write with Conflict Detection}
Het Read/Write with Conflict Detection pattern is het meest complexe waarbij verschillende end-users dezelfde data wijzigen op het moment dat de applicatie offline is. Bij dit pattern wordt er gesproken van multi-way synchronisatie waarbij een applicatie zowel de data van de server kan updaten en dat de server op zijn beurt alle andere apparaten moet updaten. Een mogelijke flow van het proces zou er als volgt kunnen uitzien:
\begin{enumerate}
\item de server database houdt alle data bij
\item de applicatie houdt lokaal een subset van de data bij die kan worden gewijzigd
\item bij synchronisatie worden de aangepaste data die lokaal wordt opgeslagen naar de server en omgekeerd
\item in de server worden de data aangepast en alle conflicting changes worden geregistreerd voor verdere behandeling
\item Vraagt de gebruiker voor conflict resolution of de applicatie kan zelf het conflict oplossen en de data aanpassen
\end{enumerate}
Een voorbeeld uit de use case van Pridiktiv die gebruik zou kunnen maken van het Read/Write with Conflict Resolution pattern is de opvolging bij wondzorg. Wanneer een end-user een bepaalde wijziging aanbrengt in het dossier van de patient, is het belangrijk dat die data niet verloren gaan indien een andere end-user ook het wondzorg dossier van een patient aanpast. Het is belangrijk om conflict resolution 'achter de schermen' op te lossen om op die manier ervoor te zorgen dat er geen aanpassingen verloren gaan.
